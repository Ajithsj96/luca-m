\section{Project organization and usage}

The project is managed by \emph{CMake} and its organized as following:

\begin{verbatim}

src
  \-->dataset      Scripts for dataset management
  \-->detector     Multiclass detector and preprocessor
  \-->facedetector Face cropping, adjusting and registration
  \-->gaborbank    Generation of gabor filters and image filtering
  \-->gui          Graphical interfaces and interaction with videos and webcam
  \-->training     2-class training and prediction
  \-->utils        String and IO utilities, CSV supports, and so on..
doc                Documentation (doxigen)
report             Report sources
resources          Containing third party resources (eg. OpenCV haar classifiers)
assets             Binary folder

\end{verbatim}

In order to configure, compile and run this project one must have:

\begin{itemize}
  \item CMake $\ge$ 2.8
  \item Opencv $\ge$ 2.4.5
  \item Python $\ge$ 2.7
\end{itemize}

Although python is not strictly required, the usage of the individual tools
without python is not recommended.

\subsection{Compilation on Linux}

Just like any other \emph{CMake} projects, the first step is to initialize the
compilation by doing:

\begin{verbatim}
mkdir build
cd build
cmake ../
\end{verbatim}

The \code{make install} command, or equivalent, compiles and moves binaries and
scripts in the \code{assets} directory.

\subsection{Compilation on Windows}

We have tested the compilation and usage on windows 8.1 using Visual Studio 12
64bit. However it should work with any version:

\begin{enumerate}
  \item Using CMake or CMakeGUI, select emotime as source folder and configure.
  \item If it complains about setting the variable \code{OpenCV\_DIR} set it to the appropriate path so that:
    \begin{enumerate}
      \item \code{C:/path/to/opencv/dir/} contains the libraries (\code{*.lib})
      \item \code{C:/path/to/opencv/dir/include} contains the include
        directories (opencv and opencv2). \textbf{IF the include directory is
        not in the right position} the project will likely not be able to
        compile due to missing reference to \code{opencv2/opencv} or similar.
    \end{enumerate}
  \item Then generate the project and compile the project \code{ALL\_BUILD}
  \item If the compilation succeeded, compile the project \code{INSTALL}
  \item Now all scripts, configuration and tools should be in the directory
    \code{assets} in the project source directory.
\end{enumerate}

\textbf{Note:} in order to execute the CLI programs, the opencv dlls must be in
the \code{PATH}.

\subsection{Usage}

The general usage is to initialize the \code{dataset} directory, fill it with
images from the \code{CK} database, train the classifiers with those images and
launch the GUI\@. Algorithm parameters can be changed editing
\code{assets/default.cfg} prior to initialization.

\subsubsection*{Initialize and fill a dataset}

\begin{verbatim}
python2 datasetInit.py [-h] dsPath config
python2 datasetFillCK.py [-h] datasetFolder cohnKanadeFolder
                                      cohnKanadeEmotionsFolder
\end{verbatim}

Any further adjustment to the configuration parameters must be
\emph{``committed''} by replacing the configuration file
\code{dataset/dataset.cfg}.

\subsubsection*{Classifier training}

You must inside a console into the \code{assest} directory, with all the
scripts and clis present.

\begin{verbatim}
python ./train_models.py --cfg dataset.cfg --mode svm --prep-train-mode 1vsAll --eye-correction /path/to/dataset
\end{verbatim}

Or also:

\begin{verbatim}
python2 datasetCropFaces.py [-h] [--eye-correction] dsFolder
python2 datasetFeatures.py [-h] dsFolder
python2 datasetPrepTrain.py [-h] [--mode {1vs1,1vsAll,1vsAllExt}] dsFolder
python2 datasetTrain.py [-h] [--mode {adaboost,svm}] dsFolder
python2 datasetVerifyPrediction.py [-h] [--mode {adaboost,svm}]
                                          [--eye-correction] dsFolder
\end{verbatim}

\subsubsection*{GUI webcam}

\begin{verbatim}
python2 gui.py [-h] [--eye-correction]
\end{verbatim}
